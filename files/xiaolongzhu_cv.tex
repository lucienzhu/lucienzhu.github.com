%%%%%%%%%%%%%%%%%%%%%%%%%%%%%%%%%%%%%%%%%%%%%%%%%%%%%%%%%%%%%%%%%%%%%%%%
%%%%%%%%%%%%%%%%%%%%%% Simple LaTeX CV Template %%%%%%%%%%%%%%%%%%%%%%%%
%%%%%%%%%%%%%%%%%%%%%%%%%%%%%%%%%%%%%%%%%%%%%%%%%%%%%%%%%%%%%%%%%%%%%%%%

%%%%%%%%%%%%%%%%%%%%%%%%%%%%%%%%%%%%%%%%%%%%%%%%%%%%%%%%%%%%%%%%%%%%%%%%
%% NOTE: If you find that it says                                     %%
%%                                                                    %%
%%                           1 of ??                                  %%
%%                                                                    %%
%% at the bottom of your first page, this means that the AUX file     %%
%% was not available when you ran LaTeX on this source. Simply RERUN  %%
%% LaTeX to get the ``??'' replaced with the number of the last page  %%
%% of the document. The AUX file will be generated on the first run   %%
%% of LaTeX and used on the second run to fill in all of the          %%
%% references.                                                        %%
%%%%%%%%%%%%%%%%%%%%%%%%%%%%%%%%%%%%%%%%%%%%%%%%%%%%%%%%%%%%%%%%%%%%%%%%

%%%%%%%%%%%%%%%%%%%%%%%%%%%% Document Setup %%%%%%%%%%%%%%%%%%%%%%%%%%%%

% Don't like 10pt? Try 11pt or 12pt
\documentclass[10pt]{article}

% This is a helpful package that puts math inside length specifications
\usepackage{calc}


% Simpler bibsection for CV sections
% (thanks to natbib for inspiration)
\makeatletter
\newlength{\bibhang}
\setlength{\bibhang}{1em}
\newlength{\bibsep}
 {\@listi \global\bibsep\itemsep \global\advance\bibsep by\parsep}
\newenvironment{bibsection}%
        {\vspace{-\baselineskip}\begin{list}{}{%
       \setlength{\leftmargin}{\bibhang}%
       \setlength{\itemindent}{-\leftmargin}%
       \setlength{\itemsep}{\bibsep}%
       \setlength{\parsep}{\z@}%
        \setlength{\partopsep}{0pt}%
        \setlength{\topsep}{0pt}}}
        {\end{list}\vspace{-.6\baselineskip}}
\makeatother

% Layout: Puts the section titles on left side of page
\reversemarginpar

%
%         PAPER SIZE, PAGE NUMBER, AND DOCUMENT LAYOUT NOTES:
%
% The next \usepackage line changes the layout for CV style section
% headings as marginal notes. It also sets up the paper size as either
% letter or A4. By default, letter was used. If A4 paper is desired,
% comment out the letterpaper lines and uncomment the a4paper lines.
%
% As you can see, the margin widths and section title widths can be
% easily adjusted.
%
% ALSO: Notice that the includefoot option can be commented OUT in order
% to put the PAGE NUMBER *IN* the bottom margin. This will make the
% effective text area larger.
%
% IF YOU WISH TO REMOVE THE ``of LASTPAGE'' next to each page number,
% see the note about the +LP and -LP lines below. Comment out the +LP
% and uncomment the -LP.
%
% IF YOU WISH TO REMOVE PAGE NUMBERS, be sure that the includefoot line
% is uncommented and ALSO uncomment the \pagestyle{empty} a few lines
% below.
%

%% Use these lines for letter-sized paper
\usepackage[paper=letterpaper,
            %includefoot, % Uncomment to put page number above margin
            marginparwidth=1.2in,     % Length of section titles
            marginparsep=.05in,       % Space between titles and text
            margin=1in,               % 1 inch margins
            includemp]{geometry}

%% Use these lines for A4-sized paper
%\usepackage[paper=a4paper,
%            %includefoot, % Uncomment to put page number above margin
%            marginparwidth=30.5mm,    % Length of section titles
%            marginparsep=1.5mm,       % Space between titles and text
%            margin=25mm,              % 25mm margins
%            includemp]{geometry}

%% More layout: Get rid of indenting throughout entire document
\setlength{\parindent}{0in}

%% This gives us fun enumeration environments. compactitem will be nice.
\usepackage{paralist}

%% Reference the last page in the page number
%
% NOTE: comment the +LP line and uncomment the -LP line to have page
%       numbers without the ``of ##'' last page reference)
%
% NOTE: uncomment the \pagestyle{empty} line to get rid of all page
%       numbers (make sure includefoot is commented out above)
%
\usepackage{fancyhdr,lastpage}
\pagestyle{fancy}
%\pagestyle{empty}      % Uncomment this to get rid of page numbers
\fancyhf{}\renewcommand{\headrulewidth}{0pt}
\fancyfootoffset{\marginparsep+\marginparwidth}
\newlength{\footpageshift}
\setlength{\footpageshift}
          {0.5\textwidth+0.5\marginparsep+0.5\marginparwidth-2in}
\lfoot{\hspace{\footpageshift}%
       \parbox{4in}{\, \hfill %
                    \arabic{page} of \protect\pageref*{LastPage} % +LP
%                    \arabic{page}                               % -LP
                    \hfill \,}}

% Finally, give us PDF bookmarks
\usepackage{color,hyperref}
\definecolor{darkblue}{rgb}{0.0,0.0,0.3}
\hypersetup{colorlinks,breaklinks,
            linkcolor=darkblue,urlcolor=darkblue,
            anchorcolor=darkblue,citecolor=darkblue}

%%%%%%%%%%%%%%%%%%%%%%%% End Document Setup %%%%%%%%%%%%%%%%%%%%%%%%%%%%


%%%%%%%%%%%%%%%%%%%%%%%%%%% Helper Commands %%%%%%%%%%%%%%%%%%%%%%%%%%%%

% The title (name) with a horizontal rule under it
%
% Usage: \makeheading{name}
%
% Place at top of document. It should be the first thing.
\newcommand{\makeheading}[1]%
        {\hspace*{-\marginparsep minus \marginparwidth}%
         \begin{minipage}[t]{\textwidth+\marginparwidth+\marginparsep}%
                {\large \bfseries #1}\\[-0.15\baselineskip]%
                 \rule{\columnwidth}{1pt}%
         \end{minipage}}

% The section headings
%
% Usage: \section{section name}
%
% Follow this section IMMEDIATELY with the first line of the section
% text. Do not put whitespace in between. That is, do this:
%
%       \section{My Information}
%       Here is my information.
%
% and NOT this:
%
%       \section{My Information}
%
%       Here is my information.
%
% Otherwise the top of the section header will not line up with the top
% of the section. Of course, using a single comment character (%) on
% empty lines allows for the function of the first example with the
% readability of the second example.
\renewcommand{\section}[2]%
        {\pagebreak[2]\vspace{1.3\baselineskip}%
         \phantomsection\addcontentsline{toc}{section}{#1}%
         \hspace{0in}%
         \marginpar{
         \raggedright \scshape #1}#2}

% An itemize-style list with lots of space between items
\newenvironment{outerlist}[1][\enskip\textbullet]%
        {\begin{itemize}[#1]}{\end{itemize}%
         \vspace{-.6\baselineskip}}

% An environment IDENTICAL to outerlist that has better pre-list spacing
% when used as the first thing in a \section
\newenvironment{lonelist}[1][\enskip\textbullet]%
        {\vspace{-\baselineskip}\begin{list}{#1}{%
        \setlength{\partopsep}{0pt}%
        \setlength{\topsep}{0pt}}}
        {\end{list}\vspace{-.6\baselineskip}}

% An itemize-style list with little space between items
\newenvironment{innerlist}[1][\enskip\textbullet]%
        {\begin{compactitem}[#1]}{\end{compactitem}}

% An environment IDENTICAL to innerlist that has better pre-list spacing
% when used as the first thing in a \section
\newenvironment{loneinnerlist}[1][\enskip\textbullet]%
        {\vspace{-\baselineskip}\begin{compactitem}[#1]}
        {\end{compactitem}\vspace{-.6\baselineskip}}

% To add some paragraph space between lines.
% This also tells LaTeX to preferably break a page on one of these gaps
% if there is a needed pagebreak nearby.
\newcommand{\blankline}{\quad\pagebreak[2]}

% Uses hyperref to link DOI
\newcommand\doilink[1]{\href{http://dx.doi.org/#1}{#1}}
\newcommand\doi[1]{doi:\doilink{#1}}

% For \url{SOME_URL}, links SOME_URL to the url SOME_URL
\providecommand*\url[1]{\href{#1}{#1}}
% Same as above, but pretty-prints SOME_URL in teletype fixed-width font
\renewcommand*\url[1]{\href{#1}{\texttt{#1}}}

% For \email{ADDRESS}, links ADDRESS to the url mailto:ADDRESS
\providecommand*\email[1]{\href{mailto:#1}{#1}}
% Same as above, but pretty-prints ADDRESS in teletype fixed-width font
%\renewcommand*\email[1]{\href{mailto:#1}{\texttt{#1}}}

%%%%%%%%%%%%%%%%%%%%%%%% End Helper Commands %%%%%%%%%%%%%%%%%%%%%%%%%%%

%%%%%%%%%%%%%%%%%%%%%%%%% Begin CV Document %%%%%%%%%%%%%%%%%%%%%%%%%%%%

\begin{document}
\makeheading{Xiaolong L. ~Zhu}

\section{Contact Information}
%
% NOTE: Mind where the & separators and \\ breaks are in the following
%       table.
%
% ALSO: \rcollength is the width of the right column of the table
%       (adjust it to your liking; default is 1.85in).
%
\newlength{\rcollength}\setlength{\rcollength}{1.85in}%
%
\begin{tabular}[t]{@{}p{\textwidth-\rcollength}p{\rcollength}}
Room 411, CYC Bldg
     & \textit{Mobile:} +852-9768-5403 \\
\href{http://www.cs.hku.hk/}%
     {Department of Computer Science}
     & \textit{MSN:} lucienzhu@hotmail.com \\
\href{http://www.hku.hk/}{The University of Hong Kong}
     & \textit{E-mail:} \email{lucienzhu@gmail.com}\\
Hong Kong, P.R.China
     & \textit{WWW:}
\href{http://xiaolongzhu.org}{www.xiaolongzhu.org}\\
\end{tabular}

%\section{Objective}
%
%Placement in an academic faculty position doing research in distributed
%systems
%\begin{innerlist}
%\item More information and auxiliary documents can be found at\\\url{http://www.tedpavlic.com/jobsearch/}
%\end{innerlist}

%\section{Citizenship}
%
%USA

\section{Research Interests}
%
Computer Vision, including \emph{Motion Capture}, \emph{Gesture Recognition}, \emph{Stereo Image Database};
Machine Learning, including \emph{Random Forest}, \emph{Boosting Methods}.

\section{Education}
%
\href{http://www.hku.hk/}{\textbf{The University of Hong Kong}},
Hong Kong, P.R.China
\begin{outerlist}

\item[] M.Phil. Candidate,
        \href{http://www.cs.hku.hk/}
             {Computer Science},
             September 2010 - present
        \begin{innerlist}
        %\item Thesis Topic: \emph{Design and Analysis of Optimal
        %    Task-Processing Agents}
        %\item Thesis Proposal: \emph{Cooperative Task Processing}
        %\item Candidacy Exam: \emph{Research
        %    Problems in Distributed Control for Energy Systems}
        \item Adviser:
              \href{http://i.cs.hku.hk/~kykwong/}
                   {Dr. Kenneth K. Y. Wong}
        \item Area of Study: Computer Vision
        \end{innerlist}

\item[] B.S.,
        \href{http://eecs.pku.edu.cn/}
             {Machine Intelligence}, September 2006 - June 2010
        \begin{innerlist}
        \item Thesis Title:  Segmentation and Classification of Range Image.
        \item \emph{Excellent Undergraduate Thesis Award}.
        \item Advisor:  \href{http://www.cis.pku.edu.cn/faculty/vision/zhaohj/index-e.htm} Dr. Huijing Zhao
        \end{innerlist}

\end{outerlist}

\section{Publications}
\begin{bibsection}
    \item \textbf{Xiaolong Zhu}, Huijing Zhao, Yiming Liu, Yipu Zhao, Hongbin Zha. Segmentation and Classification of Range Image from an Intelligent Vehicle in Urban Environment. \emph{IEEE/RSJ International Conference on Intelligent Robots and Systems (IROS)}, 2010.
    %doi:10.1109/IROS.2010.5652703

    \item Huijing Zhao, Yiming Liu,\textbf{ Xiaolong Zhu}, Yipu Zhao, Hongbin Zha. Scene Understanding in a Large Dynamic Environment through a Laser-based Sensing. \emph{IEEE International Conference on Robotics and Automation (ICRA)}, 2010.
    %doi:10.1109/ROBOT.2010.5509169
\end{bibsection}

\section{Honors \& Awards}
$\bullet$ Studentship of the University of Hong Kong, 2010-2012

$\bullet$ Top 10 Undergraduate Thesis, School of EECS in Peking University, 2010

$\bullet$ Wusi Scholarship in Peking University, 2009

$\bullet$ Outstanding Volunteer in Beijing 2008 Olympic Games, 2008

$\bullet$ First Class Honor in China Physics Olympic Games, Gansu, 2006


\section{Experience}
\href{http://www.hku.hk}{\textbf{The University of Hong Kong}}, \hfill
Hong Kong, P.R.China
\begin{outerlist}
\item[] \textit{Teaching Assistant}%
        \hfill \textbf{September 2010 to December 2010}
\begin{innerlist}
\item Assisted \href{http://i.cs.hku.hk/~eng1002b}
                    {Computer Programming and Applications}
      instructional team.
\item Provided office hours and tutorials to first-year engineering students.
\item Graded assignments on programming and drafting.
\end{innerlist}
\end{outerlist}

\blankline

\href{http://www.pku.edu.cn}{\textbf{Peking University}}, \hfill
Beijing, P.R.China
\begin{outerlist}

\item[] \textit{Undergraduate Research Assistant}%
        \hfill \textbf{September 2008 to June 2010}
\begin{innerlist}
\item Participated in the \href{http://www.poss.pku.edu.cn/}{\emph{POSS}} project, in
        \href{http://www.cis.pku.edu.cn/vision/3DVCR/3DVCR_E.html}{3D VCR Lab},
\item Analyzed range data using computer vision methods.
\end{innerlist}

\end{outerlist}

\section{Professional Experience}
%
\href{http://www.youdao.com/}{\textbf{NetEase R\&D.}}, \hfill
Beijing, P.R.China
\begin{outerlist}

\item[] \textit{Software Engineer Intern}%
        \hfill \textbf{June 2010 to August 2010}
\begin{innerlist}
\item Coded web front-end of a \href{http://bafang.163.com}{Location-based Social Network Service}
\item Cooperated with web designer.
\end{innerlist}
\end{outerlist}

\section{Technical Skills}
$\bullet$ Programming in: \texttt{Matlab}, \texttt{C/C++}, \texttt{Python}, \texttt{JavaScript}

$\bullet$ Basic Experience in: \texttt{Java}, \texttt{UNIX Shell scripting}, \texttt{HTML/CSS}, \LaTeX, \texttt{Processing}

$\bullet$ Operating Systems: Microsoft Windows family, Mac OS X, Linux Ubuntu

\end{document}

%%%%%%%%%%%%%%%%%%%%%%%%%% End CV Document %%%%%%%%%%%%%%%%%%%%%%%%%%%%%
