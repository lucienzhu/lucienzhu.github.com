%%%%%%%%%%%%%%%%%%%%%%%%%%%%%%%%%%%%%%%%%
% Medium Length Professional CV
% LaTeX Template
% Version 2.0 (8/5/13)
%
% This template has been downloaded from:
% http://www.LaTeXTemplates.com
%
% Original author:
% Trey Hunner (http://www.treyhunner.com/)
%
% Important note:
% This template requires the resume.cls file to be in the same directory as the
% .tex file. The resume.cls file provides the resume style used for structuring the
% document.
%
%%%%%%%%%%%%%%%%%%%%%%%%%%%%%%%%%%%%%%%%%

%----------------------------------------------------------------------------------------
%	PACKAGES AND OTHER DOCUMENT CONFIGURATIONS
%----------------------------------------------------------------------------------------

\documentclass{cv_professional-cn} % Use the custom resume.cls style

% set main font
\usepackage[scaled=0.92]{helvet}
\renewcommand{\familydefault}{\sfdefault}

\usepackage[resetfonts]{cmap} %解决复制问题
\usepackage[slantfont,boldfont]{xeCJK}% 允许斜体和粗体
\setCJKmainfont{SimHei}

\usepackage[left=0.75in,top=0.6in,right=0.75in,bottom=0.6in]{geometry} % Document margins
\newcommand{\specialcell}[2][c]{%
    \begin{tabular}[#1]{@{}l@{}}#2\end{tabular}}

\name{朱晓龙|个人简历} % Your name
\address{香港大学周亦卿楼411室 \\ 香港薄扶林道} % Your address
\address{\texttt{+}86\texttt{-}14714930403 \\ lucienxlzhu@gmail.com \\ xiaolongzhu.org} % Your phone number and email

% Simpler bibsection for CV sections
% (thanks to natbib for inspiration)
\makeatletter
\newlength{\bibhang}
\setlength{\bibhang}{1.5em}
\newlength{\bibsep}
 {\@listi \global\bibsep\itemsep \global\advance\bibsep by\parsep}
\newenvironment{bibsection}%
        {\vspace{\itemsep}\begin{list}{}{%
       \setlength{\leftmargin}{\bibhang}%
       %\setlength{\itemindent}{-\leftmargin}%
       \setlength{\itemsep}{\bibsep}%
       \setlength{\parsep}{\z@}%
        \setlength{\partopsep}{0pt}%
        \setlength{\topsep}{0pt}}}
        {\end{list}\vspace{\itemsep}}
\makeatother

\begin{document}

%----------------------------------------------------------------------------------------
%	Research Interest
%----------------------------------------------------------------------------------------

\begin{rSection}{兴趣方向}

{\bf 计算机视觉},包括物体检测,手势识别,图像语义标注,人脸特征点定位,场景字符识别;\\ 
{\bf 机器学习},包括随机森林,支持向量机,深度学习; \\ 
{\bf 人机交互},包括交互设计,用户研究,手势识别系统,触屏交互系统。

\end{rSection}

%----------------------------------------------------------------------------------------
%	EDUCATION SECTION
%----------------------------------------------------------------------------------------

\begin{rSection}{教育背景}

%------------------------------------------------
\begin{rSubsection}{香港大学}{2010.09 - 2015.11}{计算机科学博士研究生在读}{香港}
	\item 研究课题:彩色或深度图像中人手的检测,手型识别及位姿估计;
	\item 导师:Prof. Kenneth K.Y. Wong。
\end{rSubsection}

%------------------------------------------------
\begin{rSubsection}{北京大学}{2006.09 - 2010.06}{智能科学学士学位}{北京} 
	\item 研究课题:车载平台上的距离图像的分割与分类;
	\item 导师:赵卉菁教授;
	\item 毕业论文《距离图像的分割与分类》获信息科学技术学院本科生十佳毕业论文。
\end{rSubsection}
\end{rSection}

%----------------------------------------------------------------------------------------
%	RESEARCH EXPERIENCE SECTION
%----------------------------------------------------------------------------------------

\begin{rSection}{研究经历}
    
%------------------------------------------------
\begin{rSubsection}{香港大学计算机系}{2010.09至今}{计算机视觉组博士研究生}{香港}
	\item 利用Random Forests, RBM, SVM等机器学习模型解决彩色图及深度图上手型识别及检测问题;
	\item 合作参与人脸特征点定位和利用折射获取场景深度的理论研究及实验分析;
	\item 研究成果发表在ICCV,ACCV,ICPR等多个视觉领域国际会议上。
\end{rSubsection}

%------------------------------------------------
\begin{rSubsection}{北京大学信息科学技术学院智能科学系}{2008.09 - 2010.06}{本科生助研}{北京}
    \item 参与智能车平台研究,独立结合图像分割算法完成激光深度数据的语义标注;
    \item 进行相机标定实验,开发算法完成相机图像与深度图像的校准;
    \item 研究成果发表在ICRA,IROS等机器人领域国际会议上,毕业论文获得本科生十佳毕业论文。
\end{rSubsection}

\end{rSection}

%----------------------------------------------------------------------------------------
%	WORK EXPERIENCE SECTION
%----------------------------------------------------------------------------------------

\begin{rSection}{实习经历}
    
%------------------------------------------------
\begin{rSubsection}{联想研究院IVC研究室}{2013.06 - 2013.08}{实习研究员}{香港}
    \item 与同事探讨并参与设计触摸设备上的图像检索新模式的预研;
    \item 开发的基于热图的人脸检索项目原型展示给了联想集团CTO并获得好评。
\end{rSubsection}

%------------------------------------------------
\begin{rSubsection}{微软亚洲研究院人机交互研究组}{2012.06 - 2012.09}{实习研究员}{北京}
    \item 在HCI研究员和设计师指导下,调研并设计室内空间的手势识别系统;
    \item 独立设计完成了Kinect上基于视觉反馈的原型系统,实现手势识别的闭环设计。
\end{rSubsection}

%------------------------------------------------
\begin{rSubsection}{网易有道Mobile组}{2010.06 - 2010.08}{软件开发实习生}{北京}
    \item 负责LBS应用的前端开发,在Spring框架下利用Velocity模版语言编写Web及WAP界面;
    \item 与后端负责人和设计师紧密合作,推动项目上线并参与维护。
\end{rSubsection}

\end{rSection}

%----------------------------------------------------------------------------------------
%	Technical Skills
%----------------------------------------------------------------------------------------

\begin{rSection}{专业技能}

熟练掌握:\texttt{Matlab}, \texttt{C/C++}, \texttt{JavaScript/HTML/CSS}; \\   
了解运用:\texttt{Python}, \texttt{C\#}, \texttt{Java}, \texttt{Processing}, \texttt{UNIX Shell}; \\       
操作系统:Mac OS X, Microsoft Windows。

\end{rSection}

%----------------------------------------------------------------------------------------
%	Teaching Experience
%----------------------------------------------------------------------------------------

\begin{rSection}{教学经历}

%------------------------------------------------
\begin{rSubsection}{香港大学}{2010.09 - 2014.05}{助教}{香港}
    \item 2014年协助Dr. Kenneth K.Y. Wong的本科生课程计算机视觉,带习题课及布置作业;
    \item 2014年协助Dr. Kenneth K.Y. Wong的本科生课程计算机编程及应用,布置作业及课后答疑;
    \item 2013年协助Dr. Loretta Yi-King Choi的研究生课程可视化分析,带习题课及布置作业;
    \item 2012年协助Dr. Kenneth K.Y. Wong的本科生课程计算机视觉,带习题课及布置作业;
    \item 2011年协助Dr. Chun Kit Chui的本科生课程计算机视觉,带习题课及课后答疑;
    \item 2010年协助Dr. Kenneth K.Y. Wong的本科生课程计算机编程及应用,布置作业及课后答疑。
\end{rSubsection}

\end{rSection}

%----------------------------------------------------------------------------------------
%	TECHNICAL STRENGTHS SECTION
%----------------------------------------------------------------------------------------

\begin{rSection}{其它}

\begin{tabular}{ @{} >{\bfseries}l @{\hspace{6ex}} l }
    %-----------------------
\vspace{0.4em}荣誉奖励 & 
    \specialcell[t]{
    北京大学信息科学技术学院本科生十佳毕业论文,2010;\\
    中国RoboCup救援仿真组三等奖,2010;\\
    北京大学五四奖学金,2009;\\
    北京奥运会优秀志愿者,2008;\\
    中国甘肃物理奥林匹克竞赛一等奖,2006。
    } \\
    %-----------------------
\vspace{0.4em}课外活动 & 
\specialcell[t]{
    香港大学信息科技委员会研究生学生代表,2012-2014;\\    
    香港大学研究生会信息科技部部长,2011-2013;\\    
    北京奥运会国家体育场媒体看台专业志愿者,2008。
} \\
    %-----------------------
\vspace{0.4em}兴趣爱好 & 游泳,足球,桌游,远足。
\end{tabular}

\end{rSection}

%----------------------------------------------------------------------------------------
%	Publication List
%----------------------------------------------------------------------------------------

\begin{rSection}{论文列表}

[会议文章]

\begin{bibsection}
	\item[1.] \textbf{Xiaolong Zhu}, Xuhui Jia and Kwan-Yee K. Wong. Pixel-Level Hand Detection with Shape-aware Structured Forests. \emph{Asian Conference on Computer Vision (ACCV)}, 2014.
	
	\item[2.] \textbf{Xiaolong Zhu}, Ruoxin Sang, Xuhui Jia and Kwan-Yee K. Wong. A Hand Shape Recognizer from Simple Sketches. \emph{International Conference on Image and Vision Computing New Zealand (IVCNZ)}, 2013.
	
	\item[3.] Xuhui Jia, \textbf{Xiaolong Zhu}, Angran Lin and Kwok-Ping Chan. Face Alignment using Structured Random Regressors Combined with Statistical Shape Model Fitting. \emph{International Conference on Image and Vision Computing New Zealand (IVCNZ)}, 2013.
	
	\item[4.] \textbf{Xiaolong Zhu}, Kwan-Yee K. Wong. Single-Frame Hand Gesture Recognition Using Color and Depth Kernel Descriptors. \emph{IEEE International Conference on Pattern Recognition (ICPR)}, 2012.
	
	\item[5.] Zhihu Chen, Kwan-Yee K. Wong, Yasuyuki Matsushita, \textbf{Xiaolong Zhu}, Miaomiao Liu. Self-Calibrating Depth from Refraction. \emph{IEEE International Conference on Computer Vision (ICCV)}, 2011.
	
	\item[6.] \textbf{Xiaolong Zhu}, Huijing Zhao, Yiming Liu, Yipu Zhao, Hongbin Zha. Segmentation and Classification of Range Image from an Intelligent Vehicle in Urban Environment. \emph{IEEE/RSJ International Conference on Intelligent Robots and Systems (IROS)}, 2010.
	%doi:10.1109/IROS.2010.5652703
	
	\item[7.] Huijing Zhao, Yiming Liu,\textbf{ Xiaolong Zhu}, Yipu Zhao, Hongbin Zha. Scene Understanding in a Large Dynamic Environment through a Laser-based Sensing. \emph{IEEE International Conference on Robotics and Automation (ICRA)}, 2010.
	%doi:10.1109/ROBOT.2010.5509169
\end{bibsection}

[期刊论文]

\begin{bibsection}
	\item[1.] \textbf{Xiaolong Zhu}, Xuhui Jia and Kwan-Yee K. Wong. Structured Forests for Pixel-level Hand Detection and Hand Part Labelling. \emph{Computer Vision and Image Understanding (CVIU)}, in press.
	
    \item[2.] Zhihu Chen, Kwan-Yee K. Wong, Yasuyuki Matsushita, \textbf{Xiaolong Zhu}. Depth from Refraction Using a Transparent Medium with Unknown Pose and Refractive Index. \emph{International Journal of Computer Vision (IJCV)}, 2012.

\end{bibsection}

\end{rSection}



%----------------------------------------------------------------------------------------

\end{document}
