%%%%%%%%%%%%%%%%%%%%%%%%%%%%%%%%%%%%%%%%%
% Medium Length Professional CV
% LaTeX Template
% Version 2.0 (8/5/13)
%
% This template has been downloaded from:
% http://www.LaTeXTemplates.com
%
% Original author:
% Trey Hunner (http://www.treyhunner.com/)
%
% Important note:
% This template requires the resume.cls file to be in the same directory as the
% .tex file. The resume.cls file provides the resume style used for structuring the
% document.
%
%%%%%%%%%%%%%%%%%%%%%%%%%%%%%%%%%%%%%%%%%

%----------------------------------------------------------------------------------------
%	PACKAGES AND OTHER DOCUMENT CONFIGURATIONS
%----------------------------------------------------------------------------------------

\documentclass[UTF8]{cv_professional-cn} % Use the custom resume.cls style

% set main font
\usepackage[scaled=0.92]{helvet}
\renewcommand{\familydefault}{\sfdefault}

\usepackage[resetfonts]{cmap} %解决复制问题
\usepackage[slantfont,boldfont]{xeCJK}% 允许斜体和粗体
\setCJKmainfont{STHeiti}

\usepackage[left=0.75in,top=0.6in,right=0.75in,bottom=0.6in]{geometry} % Document margins
\newcommand{\specialcell}[2][c]{%
    \begin{tabular}[#1]{@{}l@{}}#2\end{tabular}}

\name{朱晓龙|个人简历} % Your name
\address{超参数科技(深圳)有限公司 \\ 深圳} % Your address
\address{\texttt{+}86\texttt{-}14714930403 \\ lucienxlzhu@gmail.com \\ xiaolongzhu.org} % Your phone number and email

% Simpler bibsection for CV sections
% (thanks to natbib for inspiration)
\makeatletter
\newlength{\bibhang}
\setlength{\bibhang}{1.5em}
\newlength{\bibsep}
 {\@listi \global\bibsep\itemsep \global\advance\bibsep by\parsep}
\newenvironment{bibsection}%
        {\vspace{\itemsep}\begin{list}{}{%
       \setlength{\leftmargin}{\bibhang}%
       %\setlength{\itemindent}{-\leftmargin}%
       \setlength{\itemsep}{\bibsep}%
       \setlength{\parsep}{\z@}%
        \setlength{\partopsep}{0pt}%
        \setlength{\topsep}{0pt}}}
        {\end{list}\vspace{\itemsep}}
\makeatother

\begin{document}

%----------------------------------------------------------------------------------------
%	Research Interest
%----------------------------------------------------------------------------------------

\begin{rSection}{兴趣方向}

{\bf 游戏智能},包括多智能体、生成式AI、强化学习系统、博弈论与机制设计;\\
{\bf 机器学习},包括随机森林、支持向量机、深度学习、强化学习、扩散模型、大语言模型; \\
{\bf 机器视觉},包括图像分类、物体检测、语义分割、关键点定位、多模态融合、神经辐射场;\\
{\bf 边缘计算},包括摄像头生态、并行计算、异构计算;\\
{\bf 人机交互},包括原型设计、用户研究、手势识别系统、触屏交互系统; 

\end{rSection}

%----------------------------------------------------------------------------------------
%	EDUCATION SECTION
%----------------------------------------------------------------------------------------

\begin{rSection}{教育背景}

%------------------------------------------------
\begin{rSubsection}{香港大学}{2010.09 - 2015.11}{计算机科学博士学位}{香港}
	\item 研究课题:彩色或深度图像中人手的检测,手型识别及位姿估计;
	\item 导师:Prof. Kenneth K.Y. Wong。
\end{rSubsection}

%------------------------------------------------
\begin{rSubsection}{北京大学}{2006.09 - 2010.06}{智能科学学士学位}{北京}
	\item 研究课题:车载平台上的距离图像的分割与分类;
	\item 导师:赵卉菁教授;
	\item 毕业论文《距离图像的分割与分类》获信息科学技术学院本科生十佳毕业论文。
\end{rSubsection}
\end{rSection}

%----------------------------------------------------------------------------------------
%	RESEARCH EXPERIENCE SECTION
%----------------------------------------------------------------------------------------

\begin{rSection}{工作经历}

  %------------------------------------------------
  \begin{rSubsection}{超参数科技(深圳)有限公司}{2019.03至今}{副总裁}{深圳}
    \item \textit{商业落地}:围绕 Agent、Generative AI、Distributed RL 等相关技术研发及商业项目落地;
    \item \textit{知识产权}:Game AI 相关技术的知识产权规划及申请,合作项目选题及会议论文发表;
    \item \textit{技术品牌}:机器之心AI科技年会演讲;联合清华深研院、MIT、AICrowd共同举办多智能体国际赛事Neural MMO Season 2/3;与Kaggle、Lux团队协作,举办Kaggle Lux Season 2;
    \item \textit{游戏管线}:客户端研发,美术角色、场景、动画、特效等资产制作管线管理;
    \item \textit{人才招聘}:面试超过500候选人,专业涵盖算法研究、后台开发、游戏策划、美术制作、项目管理;
    \item \textit{校企关系}:和清华深研院建立实践基地,参与人工智能实践课教学,人才培养及课题答辩;与北京大学武汉人工智能研究院项目合作;推动 VALSE、DAI 会议赞助;
    \item \textit{雇主品牌}:公司实习生纪录片策划与拍摄,牛客网校招企业宣讲全平台数据第一。
  \end{rSubsection}

  %------------------------------------------------
  \begin{rSubsection}{腾讯公司技术工程事业群AI平台部}{2016.08 至 2019.03}{高级研究员,Tech Lead}{深圳}
    \item 带领团队参与研发移动端 AI 实时计算视觉能力库;
    \item 作为 Linux Foundation DL TAC 委员,推动腾讯开源项目(如 Angel)向社区的捐献;
    \item 参与游戏 AI 项目的强化学习流程建立和模型研发;
    \item 与手机 QQ 合作,推动实时人体姿态识别的产品创新,并成功部署到 2013 年之后问世的手机;
    \item 与天天 P 图,手机 QQ 合作,推动艺术滤镜项目的落地,并参与发表 CVPR 一篇;
    \item 参与设计开发万分类主干模型,并推广应用在图像标注及嵌入场景;
    \item 对接 Tencent AI Lab 视觉中心,参与多个视觉项目原型开发及工程化落地。
  \end{rSubsection}

	%------------------------------------------------
	\begin{rSubsection}{腾讯公司技术工程事业群内部搜索部}{2015.07 - 2016.08}{基础研究工程师}{深圳}
		\item 2015 年「腾讯大咖」项目校招生;
		\item 与微信模式识别中心语音团队合作,参与了 LSTMP + CTC 语音识别模型的开发和训练;
		\item 参与「天天快报」新闻推荐项目的深度学习模块的特征工程实验;
		\item 开发了一个基于 ROS/Turtlebot 的服务机器人原型。
	\end{rSubsection}

\end{rSection}


%----------------------------------------------------------------------------------------
%	RESEARCH EXPERIENCE SECTION
%----------------------------------------------------------------------------------------

\begin{rSection}{研究经历}

%------------------------------------------------
\begin{rSubsection}{香港大学计算机系}{2010.09 - 2016.01}{计算机视觉组博士研究生}{香港}
	\item 利用 Random Forests, RBM, SVM, CNN 等模型解决彩色图及深度图上手型识别及检测问题;
	\item 合作参与人脸特征点定位和利用折射获取场景深度的理论研究及实验分析;
	\item 研究成果发表在 ICCV,WACV,ACCV,ICPR 等多个视觉领域国际会议上。
\end{rSubsection}

%------------------------------------------------
\begin{rSubsection}{北京大学信息科学技术学院智能科学系}{2008.09 - 2010.06}{本科生助研}{北京}
    \item 参与智能车平台研究,独立结合图像分割算法完成激光深度数据的语义标注;
    \item 进行相机标定实验,开发算法完成相机图像与深度图像的校准;
    \item 研究成果发表在ICRA,IROS等机器人领域国际会议上,毕业论文获得本科生十佳毕业论文。
\end{rSubsection}

\end{rSection}

%----------------------------------------------------------------------------------------
%	WORK EXPERIENCE SECTION
%----------------------------------------------------------------------------------------

\begin{rSection}{实习经历}

%------------------------------------------------
\begin{rSubsection}{联想研究院IVC研究室}{2013.06 - 2013.08}{实习研究员}{香港}
    \item 与同事探讨并参与设计触摸设备上的图像检索新模式的预研;
    \item 开发的基于热图的人脸检索项目原型展示给了联想集团CTO并获得好评。
\end{rSubsection}

%------------------------------------------------
\begin{rSubsection}{微软亚洲研究院人机交互研究组}{2012.06 - 2012.09}{实习研究员}{北京}
    \item 在HCI研究员和设计师指导下,调研并设计室内空间的手势识别系统;
    \item 独立设计完成了Kinect上基于视觉反馈的原型系统,实现手势识别的闭环设计。
\end{rSubsection}

%------------------------------------------------
\begin{rSubsection}{网易有道Mobile组}{2010.06 - 2010.08}{软件开发实习生}{北京}
    \item 负责LBS应用的前端开发,在Spring框架下利用Velocity模版语言编写Web及WAP界面;
    \item 与后端负责人和设计师紧密合作,推动项目上线并参与维护。
\end{rSubsection}

\end{rSection}


%----------------------------------------------------------------------------------------
%	Teaching Experience
%----------------------------------------------------------------------------------------

\begin{rSection}{教学经历}

%------------------------------------------------
\begin{rSubsection}{清华大学深圳研究生院}{2021.09 - 2024.08}{校外导师}{深圳}
    \item 2023年参与第八届专业实践评优大会;
    \item 2022年参与秋季人工智能实践(超参数基地)结题答辩会;
    \item 2022年参与研究生课程《人工智能技术前沿及产业应用》教学;
    \item 2021年参与研究生课程《人工智能技术前沿及产业应用》教学。
\end{rSubsection}

%------------------------------------------------
\begin{rSubsection}{香港大学}{2010.09 - 2014.05}{助教}{香港}
    \item 2014年协助Dr. Kenneth K.Y. Wong的本科生课程计算机视觉,带习题课及布置作业;
    \item 2014年协助Dr. Kenneth K.Y. Wong的本科生课程计算机编程及应用,布置作业及课后答疑;
    \item 2013年协助Dr. Loretta Yi-King Choi的研究生课程可视化分析,带习题课及布置作业;
    \item 2012年协助Dr. Kenneth K.Y. Wong的本科生课程计算机视觉,带习题课及布置作业;
    \item 2011年协助Dr. Chun Kit Chui的本科生课程计算机视觉,带习题课及课后答疑;
    \item 2010年协助Dr. Kenneth K.Y. Wong的本科生课程计算机编程及应用,布置作业及课后答疑。
\end{rSubsection}

\end{rSection}

%----------------------------------------------------------------------------------------
%	TECHNICAL STRENGTHS SECTION
%----------------------------------------------------------------------------------------

\begin{rSection}{其它}

\begin{tabular}{ @{} >{\bfseries}l @{\hspace{6ex}} l }
    %-----------------------
\vspace{0.4em}公开演讲 &
    \specialcell[t]{
    《海量智能体的算法及应用》。机器之心AI科技年会人工智能论坛,2023; \\
    《我教AI打电竞》。36氪纪录片,2022; \\
    《基于 Arm 平台的移动端 AI 应用》。Arm 开发者全球峰会,2018; \\
    《移动端 AI 的研发》。腾讯香港大学技术大咖宣讲会,2018; \\
    《人体姿态识别在移动端的应用》。腾讯 TLC 大会,2018; \\
    《Linux Foundation DL 的社区贡献》及《深度学习的发展与未来》讨论会嘉宾。LC3 中国大会,2018;\\
    《绝艺:从零开始学围棋》。腾讯 AI Lab 学术论坛,2018。
    } \\
    %-----------------------
\vspace{0.4em}荣誉奖励 &
    \specialcell[t]{
    深圳市孔雀计划 C 类人才,2017-2023;\\
    腾讯 2018 年上半年卓越研发奖,2018;\\
    腾讯 2017 年年度技术突破奖,2017;\\
    腾讯微创新奖四次,2016-2018;\\
    腾讯优秀员工,2015-2018;\\
    香港大学研究生奖学金,2010-2015;\\
    北京大学信息科学技术学院本科生十佳毕业论文,2010;\\
    中国 RoboCup 救援仿真组三等奖,2010;\\
    北京大学五四奖学金,2009;\\
    北京奥运会优秀志愿者,2008;\\
    中国甘肃物理奥林匹克竞赛一等奖,2006。
    } \\
    %-----------------------
\vspace{0.4em}社会活动 &
\specialcell[t]{
  LF DL TAC 腾讯代表,2018-2019;\\
  腾讯技术工程事业群 (TEG) 技术俱乐部发起人之一,2015-2016;\\
    香港大学信息科技委员会研究生学生代表,2012-2014;\\
    香港大学研究生会信息科技部部长,2011-2013;\\
    北京奥运会国家体育场媒体看台专业志愿者,2008。
} \\
    %-----------------------
\vspace{0.4em}专业技能 &
\specialcell[t]{
熟练掌握:\texttt{Python}, \texttt{Matlab}, \texttt{C/C++}, \texttt{JavaScript/HTML/CSS}; \\
上手经验:\texttt{C\#}, \texttt{CUDA}, \texttt{Processing}, \texttt{Obj-C},  \texttt{UNIX Shell}; \\
操作系统:Mac OS X, Linux Ubuntu, Microsoft Windows。
} \\
    %-----------------------
\vspace{0.4em}兴趣爱好 & 桌游,儿童教育,抱石,游泳,足球,远足。
\end{tabular}

\end{rSection}

%----------------------------------------------------------------------------------------

%----------------------------------------------------------------------------------------
%	Publication List
%----------------------------------------------------------------------------------------

\begin{rSection}{论文列表}

[会议文章]

\begin{bibsection}
    \item[17.]Joseph Suarez, David Bloomin, Kyoung Whan Choe, Hao Xiang Li, Ryan Sullivan, Nishaanth Kanna, Daniel Scott, Rose Shuman, Herbie Bradley, Louis Castricato, Phillip Isola, Chenghui Yu, Yuhao Jiang, Qimai Li, Jiaxin Chen, \textbf{Xiaolong Zhu}. ``Neural MMO 2.0: A Massively Multi-task Addition to Massively Multi-agent Learning." \emph{Advances in Neural Information Processing Systems (NeurIPS)}, 2024.

    \item[16.]Kai Yang, Jian Tao, Jiafei Lyu, Chunjiang Ge, Jiaxin Chen, Qimai Li, Weihan Shen, \textbf{Xiaolong Zhu}, Xiu Li. ``Using Human Feedback to Fine-tune Diffusion Models without Any Reward Model." \emph{IEEE Conference on Computer Vision and Pattern Recognition (CVPR)}, 2024.

    \item[15.]Enhong Liu, Joseph Suarez, Chenhui You, Bo Wu, Bingcheng Chen, Jun Hu, Jiaxin Chen, \textbf{Xiaolong Zhu}, Clare Zhu, Julian Togelius, Sharada Mohanty, Weijun Hong, Rui Du, Yibing Zhang, Qinwen Wang, Xinhang Li, Zheng Yuan, Xiang Li, Yuejia Huang, Kun Zhang, Hanhui Yang, Shiqi Tang, Phillip Isola. ``The NeurIPS 2022 Neural MMO Challenge: A Massively Multiagent Competition with Specialization and Trade." \emph{NeurIPS 2022 Competitions Track}, 2023.
  
    \item[14.]Yangkun Chen, Joseph Suarez, Junjie Zhang, Chenghui Yu, Bo Wu, Hanmo Chen, Hengman Zhu, Rui Du, Shanliang Qian, Shuai Liu, Weijun Hong, Jinke He, Yibing Zhang, Liang Zhao, Clare Zhu, Julian Togelius, Sharada Mohanty, Jiaxin Chen, Xiu Li, \textbf{Xiaolong Zhu}, Phillip Isola. ``Benchmarking Robustness and Generalization in Multi-Agent Systems: A Case Study on Neural MMO." \emph{International Conference on Autonomous Agents and Multiagent Systems (AAMAS)}, 2023.

    \item[13.]Hanmo Chen, Stone Tao, Jiaxin Chen, Weihan Shen, Xihui Li, Chenghui Yu, Sikai Cheng, \textbf{Xiaolong Zhu}, Xiu Li. ``Emergent Collective Intelligence from Massive-agent Cooperation and Competition." \emph{NeurIPS 2022 Deep RL Workshop}, 2023.
  
    \item[12.]Yuhao Jiang, Kunjie Zhang, Qimai Li, Jiaxin Chen, \textbf{Xiaolong Zhu}. ``Multi-Agent Path Finding via Tree LSTM." \emph{AAAI 2023 MAPF Workshop}, 2022.
  
    \item[11.]Rongqin Liang, Yuanheng Zhu, Zhentao Tang, Mu Yang, \textbf{Xiaolong Zhu}. ``Proximal Policy Optimization with Elo-based Opponent Selection and Combination with Enhanced Rolling Horizon Evolution Algorithm." \emph{IEEE Conference on Games (CoG)}, 2021.

    \item[10.] Haozhi Huang, Hao Wang, Wenhan Luo, Lin Ma, Wenhao Jiang, \textbf{Xiaolong Zhu}, Zhifeng Li, and Wei Liu. ``Real-Time Neural Style Transfer for Videos." \emph{IEEE Conference on Computer Vision and Pattern Recognition (CVPR)}, 2017.

  \item[9.] \textbf{Xiaolong Zhu}, Wei Liu, Xuhui Jia and Kwan-Yee K. Wong. ``A Two-Stage Detector for Hand Detection in Ego-Centric Videos." \emph{Winter Conference on Applications of Computer Vision (WACV)}, 2016.
  
  \item[8.] Xuhui Jia, Heng Yang, \textbf{Xiaolong Zhu}, Zhanghui Kuang, Yifeng Niu, Kwok-Ping Chan. ``Reflective Regression of 2D-3D Face Shape Across Large Pose." \emph{The British Machine Vision Conference (BMVC)}, 2016.

  \item[7.] \textbf{Xiaolong Zhu}, Xuhui Jia and Kwan-Yee K. Wong. ``Pixel-Level Hand Detection with Shape-aware Structured Forests." \emph{Asian Conference on Computer Vision (ACCV)}, 2014.

  \item[6.] \textbf{Xiaolong Zhu}, Ruoxin Sang, Xuhui Jia and Kwan-Yee K. Wong. ``A Hand Shape Recognizer from Simple Sketches." \emph{International Conference on Image and Vision Computing New Zealand (IVCNZ)}, 2013.

  \item[5.] Xuhui Jia, \textbf{Xiaolong Zhu}, Angran Lin and Kwok-Ping Chan. ``Face Alignment using Structured Random Regressors Combined with Statistical Shape Model Fitting." \emph{International Conference on Image and Vision Computing New Zealand (IVCNZ)}, 2013.

  \item[4.] \textbf{Xiaolong Zhu}, Kwan-Yee K. Wong. ``Single-Frame Hand Gesture Recognition Using Color and Depth Kernel Descriptors." \emph{IEEE International Conference on Pattern Recognition (ICPR)}, 2012.

  \item[3.] Zhihu Chen, Kwan-Yee K. Wong, Yasuyuki Matsushita, \textbf{Xiaolong Zhu}, Miaomiao Liu. ``Self-Calibrating Depth from Refraction." \emph{IEEE International Conference on Computer Vision (ICCV)}, 2011.

  \item[2.] \textbf{Xiaolong Zhu}, Huijing Zhao, Yiming Liu, Yipu Zhao, Hongbin Zha. ``Segmentation and Classification of Range Image from an Intelligent Vehicle in Urban Environment." \emph{IEEE/RSJ International Conference on Intelligent Robots and Systems (IROS)}, 2010.
  %doi:10.1109/IROS.2010.5652703

  \item[1.] Huijing Zhao, Yiming Liu,\textbf{ Xiaolong Zhu}, Yipu Zhao, Hongbin Zha. ``Scene Understanding in a Large Dynamic Environment through a Laser-based Sensing." \emph{IEEE International Conference on Robotics and Automation (ICRA)}, 2010.
  %doi:10.1109/ROBOT.2010.5509169
\end{bibsection}

[期刊论文]

\begin{bibsection}
  \item[2.] \textbf{Xiaolong Zhu}, Xuhui Jia and Kwan-Yee K. Wong. ``Structured Forests for Pixel-level Hand Detection and Hand Part Labelling." \emph{Computer Vision and Image Understanding (CVIU)}, 2015.

  \item[1.] Zhihu Chen, Kwan-Yee K. Wong, Yasuyuki Matsushita, \textbf{Xiaolong Zhu}. ``Depth from Refraction Using a Transparent Medium with Unknown Pose and Refractive Index." \emph{International Journal of Computer Vision (IJCV)}, 2012.

\end{bibsection}

\end{rSection}


%----------------------------------------------------------------------------------------
%	Patents List
%----------------------------------------------------------------------------------------

\begin{rSection}{专利列表}

\begin{bibsection}
  \item[1.] Method and apparatus for training gaze tracking model, and method and apparatus for gaze tracking. US-11797084-B2
  \item[2.] Facial expression synthesis method and apparatus, electronic device, and storage medium. US-11030439-B2
  \item[3.] Method and apparatus for training neural network model used for image processing, and storage medium. US-2019228264-A1
  \item[4.] Method and apparatus for recognizing postures of multiple persons, electronic device, and storage medium. EP-3876140-B1
  \item[5.] Method and apparatus for training pose recognition model, and method and apparatus for image recognition. US-11907848-B2
  \item[6.] Augmented reality processing method, object recognition method, and related apparatus. EP-3617995-A1
  \item[7.] Story monitoring method when robot takes elevator, electronic device, and computer storage medium. US-11242219-B2
  \item[8.] Method and apparatus for generating music. US-11301641-B2
  \item[9.] Action recognition method and apparatus, and human-machine interaction method and apparatus. US-11710351-B2
  \item[10.] Camera orientation tracking method and apparatus, device, and system. EP-3798983-B1
  \item[11.] Neural network model deployment method, prediction method, and apparatus. EP-3614316-A1
  \item[12.] Video data processing method and apparatus, and storage medium. US-11461876-B2
  \item[13.] Video image processing method and apparatus. US-10880458-B2
  \item[14.] Image recognition method and apparatus, electronic device, and readable storage medium using an update on body extraction parameter and alignment parameter. US-11417095-B2
  \item[15.] Foreground data generation method and method for applying same, related apparatus, and system. US-2021279888-A1
  \item[16.] Image processing method and apparatus. US-11200680-B2
  \item[17.] Method and apparatus for training gaze tracking model, and method and apparatus for gaze tracking. US-11797084-B2
  \item[18.] Facial expression synthesis method and apparatus, electronic device, and storage medium. US-11030439-B2
  \item[19.] Method and apparatus for training neural network model used for image processing, and storage medium. US-2019228264-A1
  \item[20.] Method and apparatus for recognizing postures of multiple persons, electronic device, and storage medium. EP-3876140-B1
  \item[21.] Method and apparatus for training pose recognition model, and method and apparatus for image recognition. US-11907848-B2
  \item[22.] Augmented reality processing method, object recognition method, and related apparatus. EP-3617995-A1
  \item[23.] Story monitoring method when robot takes elevator, electronic device, and computer storage medium. US-11242219-B2
  \item[24.] Method and apparatus for generating music. US-11301641-B2
  \item[25.] Action recognition method and apparatus, and human-machine interaction method and apparatus. US-11710351-B2
  \item[26.] Camera orientation tracking method and apparatus, device, and system. EP-3798983-B1
  \item[27.] Neural network model deployment method, prediction method, and apparatus. EP-3614316-A1
  \item[28.] Video data processing method and apparatus, and storage medium. US-11461876-B2
  \item[29.] Video image processing method and apparatus. US-10880458-B2
  \item[30.] Image recognition method and apparatus, electronic device, and readable storage medium using an update on body extraction parameter and alignment parameter. US-11417095-B2
  \item[31.] Foreground data generation method and method for applying same, related apparatus, and system. US-2021279888-A1
  \item[32.] Image processing method and apparatus. US-11200680-B2
  \item[33.] Control method and device for interactive task, storage medium and computer equipment. CN-110639208-B
  \item[34.] Model training method, model calling equipment and readable storage medium. CN-110782004-B
  \item[35.] Model training method, model using method, computer device, and storage medium. CN-111841018-B
  \item[36.] Interaction model training method, device, computer equipment and storage medium. CN-113509726-B
  \item[37.] Data sampling method and device, computer equipment and storage medium. CN-113032621-A
  \item[38.] Intelligent agent control method and device, computer equipment and storage medium. CN-112905013-A
  \item[39.] Level setting method and device, computer equipment and storage medium. CN-113134238-A
  \item[40.] Modeling method and device of intelligent agent, equipment and storage medium. CN-115645910-A
  \item[41.] Game behavior planning method, device, equipment and storage medium. CN-115624759-A
  \item[42.] Model training method, action strategy making method, server and storage medium. CN-115944924-A
  \item[43.] Video frame data sampling method and device, computer equipment and storage medium. CN-112925949-A
  \item[44.] Virtual object control method, virtual object control device, virtual object model training method, virtual object model training device and computer equipment. CN-112933605-A
  \item[45.] Intelligent agent training method, computer equipment and storage medium. CN-115759284-A
  \item[46.] Feature analysis method, device, computer equipment and storage medium. CN-113139447-B
\end{bibsection}

\end{rSection}


\end{document}
